\documentclass[11pt,a4paper]{article}

% 基础包
\usepackage[utf8]{inputenc}
\usepackage[T1]{fontenc}
\usepackage{amsmath,amssymb,amsfonts}
\usepackage{graphicx}
\usepackage{booktabs}
\usepackage{hyperref}
\usepackage{float}
\usepackage{geometry}
\usepackage{xcolor}
\usepackage{multirow}

% 页面设置
\geometry{margin=1in}

% 超链接设置
\hypersetup{
    colorlinks=true,
    linkcolor=blue,
    filecolor=magenta,
    urlcolor=cyan,
}

% 占位符命令
\newcommand{\TITLE}{{{TITLE}}}
\newcommand{\ABSTRACT}{{{ABSTRACT}}}
\newcommand{\FULLMODELRESULTS}{{{FULL_MODEL_RESULTS}}}
\newcommand{\ABLATIONTABLE}{{{ABLATION_TABLE}}}
\newcommand{\CONTRIBUTIONS}{{{CONTRIBUTIONS}}}
\newcommand{\INTERACTIONS}{{{INTERACTIONS}}}
\newcommand{\RECOMMENDATIONS}{{{RECOMMENDATIONS}}}

\title{\TITLE}
\author{Research Orchestration Cockpit\\Ablation Study Report}
\date{\today}

\begin{document}

\maketitle

\begin{abstract}
\ABSTRACT
\end{abstract}

\section{Introduction}

This report presents the results of an ablation study designed to understand the contribution of individual components to the overall model performance.

\section{Experimental Design}

\subsection{Full Model Configuration}
The full model includes all proposed components and serves as the baseline for comparison.

\subsection{Ablation Variants}
Each ablation variant removes or disables one specific component while keeping all other components unchanged.

\section{Results}

\subsection{Full Model Performance}
\FULLMODELRESULTS

\subsection{Ablation Results}
\ABLATIONTABLE

\section{Analysis}

\subsection{Component Contributions}
\CONTRIBUTIONS

\subsection{Component Interactions}
\INTERACTIONS

\section{Recommendations}
\RECOMMENDATIONS

\section{Conclusion}

The ablation study reveals the relative importance of each component in the proposed model. These findings can guide future development and optimization efforts.

\end{document}
